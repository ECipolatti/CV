% !TEX TS-program = xelatex
% !TEX encoding = UTF-8 Unicode
% -*- coding: UTF-8; -*-
% vim: set fenc=utf-8

%%%%%%%%%%%%%%%%%%%%%%%%%%%%%%%%%%%%%%%%%%%%%%%%%%%%%%%%%%%%%%%%%
%% SIMPLE-RESUME-CV
%% <https://github.com/zachscrivena/simple-resume-cv>
%% This is free and unencumbered software released into the
%% public domain; see <http://unlicense.org> for details.
%%%%%%%%%%%%%%%%%%%%%%%%%%%%%%%%%%%%%%%%%%%%%%%%%%%%%%%%%%%%%%%%%

%%%%%%%%%%%%%%%%%%%%%%%%%%%%%%%%%%%%%%%%%%%%%%%%%%%%%%%%%%%%%%%%%
%% INSTRUCTIONS FOR COMPILING THIS DOCUMENT ("CV.tex")
%% TeX ---(XeLaTeX)---> PDF:
%%
%% Method 1: Use latexmk for fully automated document generation:
%%   latexmk -xelatex "CV.tex"
%%   (add the -pvc switch to automatically recompile on changes)
%%
%% Method 2: Use XeLaTeX directly:
%%   xelatex "CV.tex"
%%   (run multiple times to resolve cross-references if needed)
%%%%%%%%%%%%%%%%%%%%%%%%%%%%%%%%%%%%%%%%%%%%%%%%%%%%%%%%%%%%%%%%%

\documentclass[letterpaper,MMMyyyy,nonstop]{simpleresumecv}
% Class options:
% a4paper, letterpaper, draft, nonstop
% MMMyyyy, ddMMMyyyy, MMMMyyyy, ddMMMMyyyy, yyyyMMdd, yyyyMM, yyyy

%%%%%%%%%%%%%%%%%%%%%%%%%%%%%%%%%%%%%%%%%%%%%%%%%%%%%%%%%%%%%%%%%
%% PREAMBLE.
%%%%%%%%%%%%%%%%%%%%%%%%%%%%%%%%%%%%%%%%%%%%%%%%%%%%%%%%%%%%%%%%%

% CV Info (to be customized).
\newcommand{\CVAuthor}{Leandro Pineda}
\newcommand{\CVTitle}{Curriculum Vitae}
\newcommand{\CVNote}{Documento actualizado el {\today}}
\newcommand{\CVWebpage}{www.linkedin.com/in/leandropineda-lp}

% PDF settings and properties.
\hypersetup{
pdftitle={\CVTitle},
pdfauthor={\CVAuthor},
pdfsubject={\CVWebpage},
pdfcreator={XeLaTeX},
pdfproducer={},
pdfkeywords={},
pdfpagemode={},
unicode=true,
bookmarks=true,
bookmarksopen=true,
pdfstartview=FitH,
pdfpagelayout=OneColumn,
pdfpagemode=UseOutlines,
hidelinks,
breaklinks}

% Shorthand.
\newcommand{\CodeCommand}[1]{\mbox{\textbf{\textbackslash{#1}}}}

%%%%%%%%%%%%%%%%%%%%%%%%%%%%%%%%%%%%%%%%%%%%%%%%%%%%%%%%%%%%%%%%%
%% ACTUAL DOCUMENT.
%%%%%%%%%%%%%%%%%%%%%%%%%%%%%%%%%%%%%%%%%%%%%%%%%%%%%%%%%%%%%%%%%

\begin{document}

%%%%%%%%%%%%%%%
% TITLE BLOCK %
%%%%%%%%%%%%%%%

\title{\CVAuthor}

\begin{subtitle}
Av. Facundo Zuviría 4016, Santa Fe, Argentina\,\SubBulletSymbol\,Edad 27 años
\par
\href{mailto:leandropineda.lp@gmail.com}
{leandropineda.lp@gmail.com}
\,\SubBulletSymbol\,
+54\,342\,4-794-999
\,\SubBulletSymbol\,
\href{\CVWebpage}
{\CVWebpage}
\end{subtitle}

\begin{body}

%%%%%%%%%%%%%%%
%% EDUCATION %%
%%%%%%%%%%%%%%%

\section
{Educación}
{Educación}
{PDF:Educación}

\textbf{Universidad Nacional del Litoral},
Santa Fe, Santa Fe, Argentina
\hfill
\DatestampY{2009} --
Actualidad

\BulletItem Facultad de Ingeniería y Ciencias Hídricas
\begin{detail}
	\SubBulletItem
	Estudiante avanzado de la carrera de grado Ingeniería en Informática
	\SubBulletItem
	Materias aprobadas: 34 \SubBulletSymbol\, Porcentaje de avance: 81\%
	\SubBulletItem Promedio con aplazos: 7.50 \SubBulletSymbol\, Promedio sin aplazos: 8.06
\end{detail}

\BigGap
\textbf{E.E.T. Nº 478 "Dr. Nicolás Avellaneda"},
Santa Fe, Santa Fe, Argentina
\hfill
\DatestampY{2005} --
\DatestampY{2008}
\BulletItem Educación Polimodal
\begin{detail}
	\SubBulletItem
	Título obtenido: Técnico en Informática Profesional y Personal.
	\SubBulletItem
	Promedio general: 8.05
\end{detail}

%\href{http://www.example.com/my-university}{\textbf{Escuela Industrial Superior}},
\BigGap
\textbf{Escuela Industrial Superior},
Santa Fe, Santa Fe, Argentina
\hfill
\DatestampY{2002} --
\DatestampY{2004}
\BulletItem
Tercer Ciclo Completo - Educación General Básica

\BigGap
\textbf{Escuela Nº 331 “Alte. Guillermo Brown”},
Santa Fe, Santa Fe, Argentina
\hfill
\DatestampY{2001}{}


\BigGap
\textbf{Colegio Don Bosco},
Bahia Blanca, Buenos Aires, Argentina
\hfill
\DatestampY{1995} --
\DatestampY{2000}

\GapNoBreak
\BulletItem
Segundo Ciclo Completo - Educación General Básica

%%%%%%%%%%%%%%%%%%%%%%%%%
%% RESEARCH EXPERIENCE %%
%%%%%%%%%%%%%%%%%%%%%%%%%

%\section
%{Research Experience}
%{Research Experience}
%{PDF:ResearchExperience}
%
%\href{http://www.example.com/my-institute}
%{\textbf{Institute for Advanced Research}},
%Science College
%
%\GapNoBreak
%\BulletItem
%Undergraduate Research Student, Science Department
%\hfill
%\DatestampYMD{2004}{05}{15} --
%\DatestampYMD{2005}{05}{15}
%\begin{detail}
%\SubBulletItem
%Project:
%Investigations on the Use of Lasers to Measure Climate Change
%\SubBulletItem
%Supervisors:
%Prof.~Jane~Citizen and
%Dr~Ann~Yone
%\SubBulletItem
%Focus:
%Climate change, lasers, statistical analysis, data analytics.
%\end{detail}

%%%%%%%%%%%%%%%%%%
%% PUBLICATIONS %%
%%%%%%%%%%%%%%%%%%

%\section
%{Publications}
%{Publications}
%{PDF:Publications}
%
%\subsection
%{Journals}
%{Journals}
%{PDF:Journals}
%
%\GapNoBreak
%\NumberedItem{[11]}
%\href{http://www.example.com/my-paper-doi-5}
%{\underline{J.~Doe}, J.~Citizen, and A.~Yone,
%``On lasers and climate change,''
%\textit{Journal of Science},
%vol.~89,
%no.~2,
%pp.~4123--4133,
%\DatestampYM{2008}{02}.}
%
%% Note the use of {\CharSpace} for aligning shorter numbers.
%\Gap
%\NumberedItem{{\CharSpace}[1]}
%\href{http://www.example.com/my-paper-doi-4}
%{\underline{J.~Doe} and J.~Citizen,
%``Measuring the extent of climate change,''
%\textit{Global Scientific Journal},
%vol.~12,
%no.~4,
%pp.~330--352,
%\DatestampYM{2006}{12}.}
%
%\BigGap
%\subsection
%{Conferences}
%{Conferences}
%{PDF:Conferences}
%
%\GapNoBreak
%\NumberedItem{[11]}
%\href{http://www.example.com/my-paper-doi-3}
%{\underline{J.~Doe}, J.~Citizen, and A.~Yone,
%``On lasers and climate change,''
%in \textit{Proceedings of the Laser Symposium},
%Las Vegas, Nevada, USA,
%\DatestampYM{2007}{01}.}
%
%\Gap
%\NumberedItem{[10]}
%\href{http://www.example.com/my-paper-doi-2}
%{A.~Yone and \underline{J.~Doe},
%``Climate change and general relativity,''
%in \textit{Proceedings of the International Astronomical Conference},
%Sydney, Australia,
%\DatestampYM{2006}{8}.}
%
%% Note the use of {\CharSpace} for aligning shorter numbers.
%\Gap
%\NumberedItem{{\CharSpace}[1]}
%\href{http://www.example.com/my-paper-doi-1}
%{\underline{J.~Doe} and J.~Citizen,
%``Measuring the extent of climate change,''
%in \textit{Proceedings of the International Climate Change Conference},
%London, UK,
%\DatestampYM{2005}{11}.}

%%%%%%%%%%%%%%%%%%%%%%%%%%%%%%%%%%%%%%%%%%%%
%% PROFESSIONAL AFFILIATIONS & ACTIVITIES %%
%%%%%%%%%%%%%%%%%%%%%%%%%%%%%%%%%%%%%%%%%%%%

%\section
%{Professional Affiliations\newline
%\& Activities}
%{Professional Affiliations \& Activities}
%{PDF:ProfessionalAffiliationsActivities}
%
%\href{http://www.example.com/my-society}
%{\textbf{Society of Professional Earth Scientists}},
%New York, USA
%
%\GapNoBreak
%\BulletItem
%Member
%\hfill
%\DatestampY{2009} --
%Present

%%%%%%%%%%%%%%%%%%%%%%%
%% CAMPUS ACTIVITIES %%
%%%%%%%%%%%%%%%%%%%%%%%

%\section
%{Campus Activities}
%{Campus Activities}
%{PDF:CampusActivities}
%
%\href{http://www.example.com/my-club}
%{\textbf{First Volunteers Club}},
%First American University
%
%\GapNoBreak
%\BulletItem
%President
%\hfill
%\DatestampYMD{2006}{08}{15} --
%\DatestampYMD{2007}{08}{15}
%\begin{detail}
%\SubBulletItem
%Lorem ipsum dolor sit amet, consectetur adipiscing elit.
%\SubBulletItem
%Curabitur vitae laoreet velit, vel ultricies est. Nam nec elit ac ante facilisis ultrices.
%\SubBulletItem
%Integer sit amet turpis dolor. Lorem ipsum dolor sit amet, consectetur adipiscing elit. Nunc at orci eu leo vulputate finibus sed et sem.
%\SubBulletItem
%Suspendisse volutpat sapien et mi cursus, gravida ornare mauris sollicitudin.
%\end{detail}

%%%%%%%%%%%%%%%%%%%%%%%%%%%
%% OTHER WORK EXPERIENCE %%
%%%%%%%%%%%%%%%%%%%%%%%%%%%

\section
{Experiencia\newline
laboral}
{ExperienciaLaboral}
{PDF:ExperienciaLaboral}

\href{http://www.sinc.unl.edu.ar}
{\textbf{Instituto de Investigación sinc(i)}},
Santa Fe, Santa Fe, Argentina

\GapNoBreak
\BulletItem
Soporte Técnico y Administrador de Sistemas
\hfill
\DatestampYM{2014}{08} --
Actualidad
\begin{detail}
\SubBulletItem
Administración de servidores GNU/Linux.
\SubBulletItem
Administración y mantenimiento de redes.
\SubBulletItem
Administración de aplicaciones basadas en arquitecturas de microservicios.
\end{detail}

%%%%%%%%%%%%%%%%%%
%% CAPACITACION %%
%%%%%%%%%%%%%%%%%%

\section
{Capacitación}
{Capacitación}
{PDF:Capacitación}

\textbf{LAN bajo MikroTik}
\hfill
\DatestampYM{2016}{10}

\BulletItem Conocer los protocolos de capa 2 y su configuración mediante Mikrotik.

\Gap
\textbf{Administración de redes informáticas mediante MikroTik}
\hfill
\DatestampYM{2015}{12}

\BulletItem Conocer las plataformas de hardware Routerboard, sus alcances y prestaciones. Administrar un nodo Mikrotik bajo diferentes modalidades. Conocer las especificaciones y requerimientos del hardware.

%%%%%%%%%%%%%%%%%%%%%%%%
%% CURSOS Y CONGRESOS %%
%%%%%%%%%%%%%%%%%%%%%%%%

\section{Cursos y \newline	Congresos}
{CursosCongresos}
{PDF:CursosCongresos}

%%%%%%%%%%%%%%%%%%%%%%%%%%%%%%%%%%%%%%%%%%%%%%%%%%%%%%%%%%%%%%%%%%%%%%%%%%%%%%%%%%%%%%%%%%%%%%%
\textbf{45º JAIIO - Jornadas Argentinas de Informática}
\hfill
\DatestampYM{2016}{09}

\BulletItem Sociedad Argentina de Informática – Centro Cultural Borges de UNTREF.
\begin{detail}
	\SubBulletItem
	Las JAIIOs se organizan como un conjunto de simposios separados, cada uno dedicado a un tema específico, de uno o dos días de duración, de tal forma de permitir la interacción de sus participantes. Se realizan sesiones paralelas donde se presentan trabajos que se publican en Anales, se discuten resultados de investigaciones y actividades sobre diferentes topicos, desarrollandose también conferencias y reuniones con la asistencia de profesionales argentinos y extranjeros.
\end{detail}
Duración: 3 días. \SubBulletSymbol\, Buenos Aires, Argentina.

\Gap
%%%%%%%%%%%%%%%%%%%%%%%%%%%%%%%%%%%%%%%%%%%%%%%%%%%%%%%%%%%%%%%%%%%%%%%%%%%%%%%%%%%%%%%%%%%%%%%
\textbf{Introduction to Application Security - Microsoft Research}
\hfill
\DatestampYM{2016}{07}

\BulletItem Escuela de Ciencias Informáticas – Departamento de Computación, UBA.
\begin{detail}
	\SubBulletItem
	This course will introduce students to adversarial mindset, i.e., how to think like an attacker, and will include an overview of the most common types of vulnerabilities in use in exploitation today, including buffer overruns and the most common web vulnerabilities such as cross-site scripting and SQL injection.
	\SubBulletItem
	Temáticas: Introduction to the adversarial mindset. Memory safety and buffer overruns. Web application security. Privacy. Tools for static and runtime analysis. Hacking for fun and profit: case studies: cars, routers, and other devices.
	\SubBulletItem
	Docente: Ben Livshits.
	\SubBulletItem
	Calificación obtenida: 8 (ocho).
\end{detail}
Duración: 5 días. \SubBulletSymbol\, Buenos Aires, Argentina.

\Gap
%%%%%%%%%%%%%%%%%%%%%%%%%%%%%%%%%%%%%%%%%%%%%%%%%%%%%%%%%%%%%%%%%%%%%%%%%%%%%%%%%%%%%%%%%%%%%%%
\textbf{Internet de las Cosas (IoT) - Cablevisión}
\hfill
\DatestampYM{2016}{07}

\BulletItem Escuela de Ciencias Informáticas – Departamento de Computación, UBA.
\begin{detail}
	\SubBulletItem
	Objetivo: mostrar a que apunta el IoT partiendo de lo que se puede hacer hoy, dando ejemplos prácticos y  mostrando que tiene que ofrecer la tecnología para que la gente	esté dispuesta a pagar por estos servicios.	
	\SubBulletItem
	Docente: Gabriel Carro - Departamento de I+D.
\end{detail}
Duración: 5 días. \SubBulletSymbol\, Buenos Aires, Argentina.

\Gap
%%%%%%%%%%%%%%%%%%%%%%%%%%%%%%%%%%%%%%%%%%%%%%%%%%%%%%%%%%%%%%%%%%%%%%%%%%%%%%%%%%%%%%%%%%%%%%%
\textbf{Principios de cyber-security para entornos corporativos - Intel Security}
\hfill
\DatestampYM{2016}{07}

\BulletItem Escuela de Ciencias Informáticas – Departamento de Computación, UBA.
\begin{detail}
	\SubBulletItem
	Objetivo: presentar como se estructura la seguridad de una empresa a los fines de proteger y defender la  misma. También se discutirán algunas tácticas, técnicas y procedimientos (TTPs) conocidos que están siendo utilizados en ciber-ataques.
	\SubBulletItem
	Temáticas: Anatomía de un ataque y el \textit{Cyber Kill Chain}. Seguridad en redes e infraestructura de una empresa. Defensa en profundidad. Protegiendo redes corporativas. Protegiendo \textit{endpoints} corporativos.
	\SubBulletItem
	Docentes: Matías Cuenca-Acuna, Leonardo Frittelli, Marcelo Lorenzati,  María  Emilia  Torino  y Gustavo Yaguez.
	\SubBulletItem
	Calificación obtenida: 10 (diez).
\end{detail}
Duración: 5 días. \SubBulletSymbol\, Buenos Aires, Argentina.

\Gap
%%%%%%%%%%%%%%%%%%%%%%%%%%%%%%%%%%%%%%%%%%%%%%%%%%%%%%%%%%%%%%%%%%%%%%%%%%%%%%%%%%%%%%%%%%%%%%%
\textbf{VI Congreso Iberoamericano de Investigadores y Docentes de Derecho e Informática}
\hfill
\DatestampYM{2016}{05}

\BulletItem Facultad de Ingeniería y Ciencias Hídricas – Universidad Nacional del Litoral.
\begin{detail}
	\SubBulletItem
	El Congreso Iberoamericano de Docentes e Investigadores en Derecho e Informática, es un encuentro que convoca a Docentes e Investigadores de la relación entre el Derecho y la Informática, con la intención de generar un ámbito que permita difundir e impulsar el avance en la investigación, generar lazos de cooperación, y profundizar el conocimiento a partir del debate, y el intercambio de ideas, agregando valor a los esfuerzos individuales.
\end{detail}
Duración: 3 días. \SubBulletSymbol\, Santa Fe, Argentina.

\Gap
%%%%%%%%%%%%%%%%%%%%%%%%%%%%%%%%%%%%%%%%%%%%%%%%%%%%%%%%%%%%%%%%%%%%%%%%%%%%%%%%%%%%%%%%%%%%%%%
\textbf{El rol de la visión en el emprendimiento}
\hfill
\DatestampYM{2015}{02}

\BulletItem Secretaría de Vinculación Tecnológica y Desarrollo Productivo – Universidad Nacional del Litoral.
\begin{detail}
	\SubBulletItem
	Objetivo: Identificar la importancia de la misión y visión para el éxito del emprendimiento, brindando las principales técnicas y herramientas que facilitan su materialización.
	\SubBulletItem
	Temáticas: Características del emprendedor. La misión del emprendimiento. La visión de futuro del emprendedor. Llevando adelante la Misión y Visión: el Plan de Negocios.
	\SubBulletItem
	Docentes: Gustavo Miazzi y Hugo Amante. 
\end{detail}
Duración: 6 horas. \SubBulletSymbol\, Santa Fe, Argentina.

%%%%%%%%%%%%%%%
%% LANGUAGES %%
%%%%%%%%%%%%%%%

\section
{Lenguajes}
{Lenguajes}
{PDF:Lenguajes}

\BulletItem
Ingles: Nivel medio.

\GapNoBreak
\BulletItem
Español: Nativo.

%%%%%%%%%%%%
%% SKILLS %%
%%%%%%%%%%%%

\section
{Habilidades}
{Habilidades}
{PDF:Habilidades}

\textbf{Programación}
\BulletItem
Conocimientos avanzados: Python \SubBulletSymbol\, C++
\BulletItem
Otros lenguajes: Java \SubBulletSymbol\, Scala \SubBulletSymbol\, SQL \SubBulletSymbol\, PHP

\textbf{Tecnologías}
\BulletItem Apache Spark \SubBulletSymbol\, Apache Flink \SubBulletSymbol\, Apache Kafka \SubBulletSymbol\, Docker

%%%%%%%%%%%%%%%
%% INTERESTS %%
%%%%%%%%%%%%%%%

\section
{Intereses}
{Intereses}
{PDF:Intereses}

Seguridad Informática, Machine Learning, Redes de Información, Big Data, Infraestructura.

	

%%%%%%%%%%%%%%%%%
%%% REFERENCES %%
%%%%%%%%%%%%%%%%%
%
%\section
%{References}
%{References}
%{PDF:References}
%
%\BulletItem
%\textbf{Professor Jonathan Public}
%\newline
%Professor of Geology and Mechanical Engineering
%\newline
%First American University
%\newline
%1000 First Avenue, Springfield, Massachusetts 22222, USA
%\newline
%\href{mailto:jonathanpublic@example.com}
%{jonathanpublic@example.com}
%\,\SubBulletSymbol\,
%+1\,(555)\,222-2222
%
%\BigGap
%\BulletItem
%\textbf{Dr Alice Bob Carol}
%\newline
%Director, Research \& Development
%\newline
%Alpha Engineering Firm
%\newline
%20 North Street, Oakland, Ohio 33333, USA
%\newline
%\href{mailto:alicebobcarol@example.com}
%{alicebobcarol@example.com}
%\,\SubBulletSymbol\,
%+1\,(555)\,333-3333
%
%%%%%%%%%%%%%%%%%%%%%%%%%%%%%%%%%%
%%% SECTION WITH USAGE EXAMPLES %%
%%%%%%%%%%%%%%%%%%%%%%%%%%%%%%%%%%
%
%\section
%{Section\newline
%With\newline
%Usage\newline
%Examples}
%{Section With Usage Examples (For PDF Bookmark)}
%{PDF:SectionWithUsageExamples:ForPDFLink}
%
%\subsection
%{This is a Subsection}
%{This is a Subsection}
%{PDF:ThisIsASubSection}
%
%\GapNoBreak
%\BulletItem
%Use \CodeCommand{section} and \CodeCommand{subsection} to create sections and subsections.
%These will appear in the PDF bookmarks too.
%
%\GapNoBreak
%\BulletItem
%This is the second \CodeCommand{BulletItem}.
%Long items are automatically indented.
%Lorem ipsum dolor sit amet, consectetur adipiscing elit.
%Sed sed aliquam massa.
%\begin{detail}
%\SubBulletItem
%This is a \CodeCommand{SubBulletItem}.
%Long items are automatically indented.
%Lorem ipsum dolor sit amet, consectetur adipiscing elit.
%Sed sed aliquam massa.
%Aliquam dignissim mi non enim feugiat elementum.
%Donec sit amet turpis ac velit ultrices volutpat.
%Aliquam vitae elit massa.
%\SubBulletItem
%This is the second \CodeCommand{SubBulletItem}.
%\SubBulletItem
%The \CodeCommand{SubBulletItem}'s are between
%\CodeCommand{begin\{detail\}} and
%\CodeCommand{end\{detail\}} so that they are typeset in a smaller font.
%\end{detail}
%
%\Gap
%\BulletItem
%This is the third \CodeCommand{BulletItem}.
%
%\Gap
%\BulletItem
%A \CodeCommand{Gap} or \CodeCommand{GapNoBreak} is inserted between the \CodeCommand{BulletItem}'s so that there is a small vertical space between them.
%The ``NoBreak'' version prevents page breaking, and should be used to avoid orphaned headings and other formatting issues.
%
%\BigGap
%\subsection
%{This is the Second Subsection}
%{This is the Second Subsection}
%{PDF:ThisIsTheSecondSubSection}
%
%\GapNoBreak
%\BulletItem
%A \CodeCommand{BigGap} or \CodeCommand{BigGapNoBreak} is inserted between subsections so that there is a bigger vertical space between them.
%The ``NoBreak'' version prevents page breaking.
%
%%%%%%%%%%%%%%%%%%%%%%%%%%%%%%%%%%%%%%%%%%
%%% ANOTHER SECTION WITH USAGE EXAMPLES %%
%%%%%%%%%%%%%%%%%%%%%%%%%%%%%%%%%%%%%%%%%%
%
%\section
%{Another\newline
%Section\newline
%With\newline
%Usage\newline
%Examples}
%{Another Section With Usage Examples (For PDF Bookmark)}
%{PDF:AnotherSectionWithUsageExamples:ForPDFLink}
%
%\textbf{This is a Plain Heading},
%followed by an \CodeCommand{hfill} and a date range
%\hfill
%\DatestampYM{2015}{10} --
%\DatestampYM{2015}{12}
%
%\GapNoBreak
%\BulletItem
%This is a \CodeCommand{BulletItem}.
%\begin{detail}
%\SubBulletItem
%This is a \CodeCommand{SubBulletItem}.
%\end{detail}
%
%\GapNoBreak
%\BulletItem
%This is a \CodeCommand{BulletItem}.
%\begin{detail}
%\SubItem
%This is a \CodeCommand{SubItem}, which has no bullet.
%Note the alignment with the \CodeCommand{BulletItem} above.
%\end{detail}
%
%\GapNoBreak
%\Item
%This is an \CodeCommand{Item}, which has no bullet.
%Note the alignment with the \CodeCommand{BulletItem} above.
%\begin{detail}
%\SubItem
%This is a \CodeCommand{SubItem}.
%\end{detail}
%
%\GapNoBreak
%\NumberedItem{[16]}
%This is a \CodeCommand{NumberedItem}.
%Note the alignment with the \CodeCommand{SubBulletItem} above.
%
%\GapNoBreak
%\NumberedItem{{\CharSpace}[6]}
%This is a \CodeCommand{NumberedItem} with a \CodeCommand{CharSpace} in its argument for padding shorter numbers.
%Note the alignment with the \CodeCommand{NumberedItem} above.
%
%\BigGap
%\textbf{Usage Notes}
%
%\GapNoBreak
%\BulletItem
%New Lines and Paragraphs
%\begin{detail}
%\SubBulletItem
%To create a new line within the same paragraph (i.e., with the same indentation), use \CodeCommand{newline} instead of \CodeCommand{\textbackslash}.
%The latter will not work because it breaks the long table.
%\SubBulletItem
%To create a new paragraph, use \CodeCommand{par} or simply leave an empty line.
%Paragraph indentations (from
%\CodeCommand{Item},
%\CodeCommand{SubItem},
%\CodeCommand{BulletItem},
%\CodeCommand{SubBulletItem},
%etc.) do not carry across different paragraphs.
%\end{detail}
%
%\Gap
%\BulletItem
%Vertical Spacing Between Items
%\begin{detail}
%\SubBulletItem
%Use \CodeCommand{Gap} or \CodeCommand{GapNoBreak} to insert a small vertical space between items within the same section.
%\SubBulletItem
%Use \CodeCommand{BigGap} or \CodeCommand{BigGapNoBreak} to insert a bigger vertical space between items within the same section.
%\SubBulletItem
%The ``NoBreak'' versions prevent page breaking.
%\end{detail}
%
%\Gap
%\BulletItem
%Dates
%\begin{detail}
%\SubBulletItem
%Use
%\CodeCommand{DatestampYMD\{YYYY\}\{MM\}\{DD\}},
%\CodeCommand{DatestampYM\{YYYY\}\{MM\}}, and
%\CodeCommand{DatestampY\{YYYY\}}
%to specify dates.
%\SubBulletItem
%Change the date format option passed to the document class to adjust how dates are displayed throughout the document:
%MMMyyyy (``Dec~2010''),
%ddMMMyyyy (``31~Dec~2010''),
%MMMMyyyy (``December~2010''),
%ddMMMMyyyy (``31~December~2010''),
%yyyyMMdd (``2010-12-31''),
%yyyyMM (``2010-12''),
%yyyy (``2010'').
%\end{detail}
%
%%%%%%%%%%%%%%%%%%%%%%%%%%%%%%%%%%%%
%%% MULTILINGUAL UNICODE EXAMPLES %%
%%%%%%%%%%%%%%%%%%%%%%%%%%%%%%%%%%%%
%
%\section
%{Multilingual Unicode Examples}
%{Multilingual Unicode Examples}
%{PDF:MultilingualUnicodeExamples}
%
%\BulletItem
%Assortment of unicode characters from
%\href{http://www.ltg.ed.ac.uk/~richard/unicode-sample.html}
%{http://www.ltg.ed.ac.uk/{\TildeSymbol}richard/unicode-sample.html}
%
%\begin{detail}
%\SubItem
%\textbf{Latin Extended-A}
%Ā ā Ă ă Ą ą Ć ć Ĉ ĉ Ċ ċ Č č Ď ď Đ đ Ē ē Ĕ ĕ Ė ė Ę ę Ě ě Ĝ ĝ Ğ ğ Ġ ġ Ģ ģ Ĥ ĥ Ħ ħ Ĩ ĩ Ī ī Ĭ ĭ Į į İ ı IJ ij Ĵ ĵ
%\textbf{Latin Extended-B}
%ƀ Ɓ Ƃ ƃ Ƅ ƅ Ɔ Ƈ ƈ Ɖ Ɗ Ƌ ƌ ƍ Ǝ Ə Ɛ Ƒ ƒ Ɠ Ɣ ƕ Ɩ Ɨ Ƙ ƙ ƚ ƛ Ɯ Ɲ ƞ Ɵ Ơ ơ Ƣ ƣ Ƥ ƥ Ʀ Ƨ ƨ Ʃ ƪ ƫ Ƭ ƭ Ʈ Ư ư Ʊ Ʋ Ƴ ƴ Ƶ
%\textbf{Latin Extended Additional}
%Ḁ ḁ Ḃ ḃ Ḅ ḅ Ḇ ḇ Ḉ ḉ Ḋ ḋ Ḍ ḍ Ḏ ḏ Ḑ ḑ Ḓ ḓ Ḕ ḕ Ḗ ḗ Ḙ ḙ Ḛ ḛ Ḝ ḝ Ḟ ḟ Ḡ ḡ Ḣ ḣ Ḥ ḥ Ḧ ḧ Ḩ ḩ Ḫ ḫ Ḭ ḭ Ḯ ḯ Ḱ ḱ Ḳ ḳ Ḵ ḵ
%\textbf{Greek}
%ʹ ͵ ͺ ; ΄ ΅ Ά · Έ Ή Ί Ό Ύ Ώ ΐ Α Β Γ Δ Ε Ζ Η Θ Ι Κ Λ Μ Ν Ξ Ο Π Ρ Σ Τ Υ Φ Χ Ψ Ω Ϊ Ϋ ά έ ή ί ΰ α β γ δ ε ζ η θ
%\textbf{Cyrillic}
%Ё Ђ Ѓ Є Ѕ І Ї Ј Љ Њ Ћ Ќ Ў Џ А Б В Г Д Е Ж З И Й К Л М Н О П Р С Т У Ф Х Ц Ч Ш Щ Ъ Ы Ь Э Ю Я а б в г д е ж з
%\textbf{Hebrew}
%א ב ג ד ה ו ז ח ט י ך כ ל ם מ ן נ ס ע ף פ ץ צ ק ר ש ת װ ױ ײ ֝ ֞ ֟ ֠ ֡ ֣ ֤ ֥ ֦ ֧ ֨ ֩ ֪ ֫ ֬ ֭ ֮ ֯ ְ ֱ ֒ ֓ ֔
%\textbf{Armenian}
%{\UseSecondaryFont
%Ա Բ Գ Դ Ե Զ Է Ը Թ Ժ Ի Լ Խ Ծ Կ Հ Ձ Ղ Ճ Մ Յ Ն Շ Ո Չ Պ Ջ Ռ Ս Վ Տ Ր Ց Ւ Փ Ք Օ Ֆ ՙ ՚ ՛ ՜ ՝ ՞ ՟ ա բ գ դ ե զ}
%\textbf{Thai}
%{\UseSecondaryFont
%ก ข ฃ ค ฅ ฆ ง จ ฉ ช ซ ฌ ญ ฎ ฏ ฐ ฑ ฒ ณ ด ต ถ ท ธ น บ ป ผ ฝ พ ฟ ภ ม ย ร ฤ ล ฦ ว ศ ษ ส ห ฬ อ ฮ ฯ ะ ั า ำ ิ}
%\end{detail}
%
\end{body}

%%%%%%%%%%%
% CV NOTE %
%%%%%%%%%%%

\UseNoteFont%
\null\hfill%
[\textit{\CVNote}]%
\hspace{2.0mm}\null

\end{document}
